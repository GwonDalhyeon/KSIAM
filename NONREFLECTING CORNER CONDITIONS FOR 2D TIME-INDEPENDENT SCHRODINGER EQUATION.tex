\documentclass{beamer}%[slidestop,compress,mathserif]
\usetheme{default}
\usepackage{beamerthemeshadow}
%\usetheme{default}
\logo{\includegraphics[height=0.7cm]{Llogo.png}}
\usetheme{Ilmenau} % Beamer theme v 3.0
\useoutertheme{miniframes}
%\useoutertheme{default}
\usepackage{amsfonts,amssymb,amsmath,mathrsfs,graphicx}
\setbeamercolor{frametitle}{fg=black,bg=white}
%\usepackage{subfigure}
\usefonttheme[onlysmall]{structurebold}

\begin{document}

\title{NONREFLECTING CORNER CONDITIONS FOR 2D TIME-INDEPENDENT SCHRODINGER EQUATION}
\institute{Seoul National University}
\author{Han Heejae, Gwon Dalhyeon,\\ Kim Hanchul and Kang Myungjoo}
\date{Nov 19th, 2013}
\frame{\titlepage}
%\frame{\frametitle{Table of contents}\tableofcontents}
\frame{\tableofcontents}


%%%%%%%%%%%%%%%%%%%%%%%%%%%%%%%%%%%%%%%%%%%%%%%%%%%%%%%%%%%%%%%
%%%%%%%%%%%%%%%%%%%%%%%% section 1 %%%%%%%%%%%%%%%%%%%%%%%%%%%%
%%%%%%%%%%%%%%%%%%%%%%%%%%%%%%%%%%%%%%%%%%%%%%%%%%%%%%%%%%%%%%%
\section{Introduction}
% page1
\subsection{}
\frame
{
%\frametitle{Linear Advection Equation}
\small 

\begin{block}{The wave equation}
$$\dfrac{\partial^2 w}{\partial t^2}-\dfrac{\partial^2 w}{\partial x^2}-\dfrac{\partial^2 w}{\partial y^2}=0$$
\end{block}
$\circ$ We derive \textcolor{red}{non-reflecting boundary conditions} on the wall $y=0$ for wave equation in $t,~y \geq 0$\\

$\circ$ 2D wave solutions traveling to the left are
$$w= \exp(i(k_1 x + \omega t + \sqrt{\omega^2-k_1^2}y)), ~ \omega^2-k_1^2 >0, ~\omega>0  $$
$\circ$ $\omega$ : the interpretation of frequency\\
$\circ$ $\dfrac{k_1}{\omega}=\sin\theta$, where $\theta$ is the angle of incidence of the wave upon the boundary $y=0$.
}


% page2
\frame
{
\small
$\circ$ For fixed $\omega, k_1$, boundary condition is given by a zero coefficient
$$\left(\dfrac{d}{dy}-i\sqrt{\omega^2-k_1^2}\right)w|_{y=0}=0$$

$\circ$ In general, 
$$w(x,y,t)=\iint_{\omega^2-k_1^2 >0}\exp(i(k_1 x + \omega t + \sqrt{\omega^2-k_1^2}y))\hat{w}(k_1, 0,  \omega)d\omega dk_1$$

\Rightarrow ~
$\left(\dfrac{\partial}{\partial y}-\sqrt{\dfrac{\partial^2}{\partial t^2}-\dfrac{\partial^2}{\partial x^2}}\right)w|_{y=0}=0$, where
$\displaystyle \sqrt{\dfrac{\partial^2}{\partial t^2}-\dfrac{\partial^2}{\partial x^2}}=\iint \sqrt{\omega^2-k_1^2}\exp(i(k_1 x + \omega t))\hat{w}(k_1, 0,  \omega)d\omega dk_1$

}


% page3
\frame
{
\frametitle{Pade approxiamation}
\small
$\circ$ Pade approximations of the function $z \rightarrow \sqrt{1-z}$ are
\begin{itemize}
  \item (1st) $\sqrt{1-z}=1+O(x)$
  \item (2nd) $\sqrt{1-z}=1-\dfrac{1}{2}z+O(z^2)$
  \item (3rd) $\sqrt{1-z}=1-\dfrac{z}{2-z/2}+O(z^3)$
\end{itemize}


}

% page4
\frame
{
%\frametitle{}
\small
\quad We observe that under Fourier transform
$$i\omega \leftrightarrow \dfrac{\partial}{\partial t}, ~ik_1 \leftrightarrow \dfrac{\partial}{\partial x}$$
\quad By Pade approximations, absorbing boundary conditions for the wave equation on the wall $y=0$:\newline
From $\left(\dfrac{\partial}{\partial y}-\sqrt{\dfrac{\partial^2}{\partial t^2}-\dfrac{\partial^2}{\partial x^2}}\right)w|_{y=0}=0$,
\begin{itemize}
  \item (1st approx.) $w_y-w_t|_{y=0}=0$
  \item (2nd approx.) $w_{yt}-w_{tt}+\dfrac{1}{2}w_{xx}|_{y=0}=0$
  \item (3rd approx.) $w_{ytt}-w_{ttt}-\dfrac{1}{4}w_{yxx}+\dfrac{3}{4}w_{txx}|_{y=0}=0$
\end{itemize}
}

% page5
\frame
{
%\frametitle{}
\small
\quad However, these conditions do not always give satisfactory solutions for
\begin{enumerate}
  \item long time simulations
  \item , or when the size of the computational box is small with respect to the wavelength.
\end{enumerate}\\
$\Rightarrow$ \textcolor{red}{Increasing the order of the approximation.}

\vspace{10}
\quad Note that $\sqrt{1-z}\approx \left( 1- \displaystyle \sum_{m=1}^{L} \beta_m \dfrac{z}{1-\alpha_m z} \right) \text{ for } z\in[0,1]$, with appropriate $L, ~\beta_m,~ \text{and}~\alpha_m$.\\
\quad We choose the Pade coefficients,
$$\beta_m=\dfrac{2}{2L+1}\sin^2\left(\dfrac{m\pi}{2L+1}\right), \alpha_m=\cos^2\left(\dfrac{m\pi}{2L+1} \right).$$

%\quad This approximation stability has been proved using Kreiss's criterion for
%$$ 0\leq\alpha_m<1,~\beta_m\geq0,~\text{and}~ \sum_{m=1}^{L}\dfrac{\beta_m }{1-\alpha_m z}<1 .$$


}


% page6
\frame
{
%\frametitle{}
\small
$$\mathcal{F}\left(\dfrac{\partial u}{\partial y}\right)+i\omega\left(1-\sum_{m=1}^{L}\beta_m \dfrac{k_1^2}{\omega^2-\alpha_m k_1^2}\right)\mathcal{F}u=0~ \text{on}~y=0$$

Something...

}


%%%%%%%%%%%%%%%%%%%%%%%%%%%%%%%%%%%%%%%%%%%%%%%%%%%%%%%%%%
%%%%%%%%%%%%%%%%%%%%%%% Section 2 %%%%%%%%%%%%%%%%%%%%%%%%
%%%%%%%%%%%%%%%%%%%%%%%%%%%%%%%%%%%%%%%%%%%%%%%%%%%%%%%%%%
\section{The High Order ABC}
\subsection{}
% page1
\frame
{
%\frametitle{}
\small
\quad Consider the way for Fourier invering
$$\mathcal{F}\left(\dfrac{\partial u}{\partial y}\right)+i\omega\left(1-\sum_{m=1}^{L}\beta_m \dfrac{k_1^2}{\omega^2-\alpha_m k_1^2}\right)\mathcal{F}u=0$$
into the time space domain based on the use of \textcolor{red}{auxiliary functions}.\\
\vspace{10}
\quad We introduce the function for each $l=1,\cdots,L$
\begin{center}
$\mathcal{F}\varphi_l^{(2)}(k_1,\omega)=\dfrac{k_1^2}{\omega^2-\alpha_1k_1^2}\mathcal{F}u(k_1,0,\omega).$\\
\end{center}

\quad Equation become
\begin{block}{}
\begin{center}
$\dfrac{d\mathcal{F}u}{d y}+i\omega \mathcal{F}u- \sum_{l=1}^{L}i\omega\beta_l \mathcal{F}\varphi_l^{(2)}=0 \\
(\omega^2-\alpha_1k_1^2)\mathcal{F}\varphi_l^{(2)}(k_1,\omega)=k_1^2 \mathcal{F}u(k_1,0,\omega)$
\end{center}
\end{block}
}


% page2
\frame{
\small
\quad Applying the Fourier transform, we obtain ABC formulation:
\begin{block}{}
\begin{center}
$\dfrac{\partial u}{\partial y}(x,0,t)+\dfrac{\partial u}{\partial t}(x,0,t)- \sum_{m=1}^{L}\beta_l \dfrac{\varphi_m^{(2)}}{\partial t}(x,t)=0 \\
\dfrac{\partial^2\varphi_l^{(2)}}{\partial t^2}(x,t)-\dfrac{\partial^2\psi_l^{(2)}}{\partial x^2}(x,t)=0~ 
\text{with}~ \psi_l^{(2)}(x,t)=\alpha_l \varphi_l^{(2)}(x,t)+u(x,0,t)$
\end{center}
\end{block}

$\circ$ Remark. Blah Blah...

}



% page3
\frame{
\frametitle{Interim Summary}
\small

}



%%%%%%%%%%%%%%%%%%%%%%%%%%%%%%%%%%%%%%%%%%%%%%%%%%%%%%%%%%
%%%%%%%%%%%%%%%%%%%%%%% Section 3 %%%%%%%%%%%%%%%%%%%%%%%%
%%%%%%%%%%%%%%%%%%%%%%%%%%%%%%%%%%%%%%%%%%%%%%%%%%%%%%%%%%
\section{The Corner Problem}

\subsection{}
% page1
\frame{
\frametitle{Analysis the Question with Corner}
\small
\quad Let set the problem in $\Omega=\{x<0,y<0\}$.\\
\quad $u$ is satisfied a previous equation on $\{x=0,y<0\}$ and a below equation on $\{x<0,y=0\}$:

\begin{block}{}
\begin{center}
$\dfrac{\partial u}{\partial x}(0,y,t)+\dfrac{\partial u}{\partial t}(0,y,t)- \sum_{m=1}^{L}\beta_l \dfrac{\varphi_m^{(1)}}{\partial t}(y,t)=0 \\
\dfrac{\partial^2\varphi_l^{(1)}}{\partial t^2}(y,t)-\dfrac{\partial^2\psi_l^{(1)}}{\partial y^2}(y,t)=0~ 
\text{with}~ \psi_l^{(1)}(y,t)=\alpha_l \varphi_l^{(1)}(y,t)+u(0,y,t)$
\end{center}
\end{block}
$\circ$ But, there is some problem.
\begin{enumerate}
  \item Both equation are not enough to compute in $\Omega$.
  \item Without additional condition, the second derivative with the auxilary functions at corner cannot be computed.
\end{enumerate}\\

}

% page2
\frame{
%\frametitle{}
\small
Blah Blah\\
$$\dfrac{3}{2}\cdot \dfrac{\partial u}{\partial t}(0,0,t)+\dfrac{\partial u}{\partial x}(0,0,t)\dfrac{\partial u}{\partial y}(0,0,t)=0$$


}

% page3
\frame{
%\frametitle{}
\small

\quad Add two simple reflections on each boundary and one double reflection.\\
\quad Let
\begin{eqnarray*}
u(x,y,t)&=&\left(exp(ik_1x)+R^{(1)}exp(-ik_1x)\right)\\
&&\qquad\qquad \left(exp(ik_2y)+R^{(2)}exp(-ik_2y)\right)exp(-i\omega t)
\end{eqnarray*}
\begin{eqnarray*}
\varphi_l^{(1)}(y,t)=\Phi_l^{(1)}\cdot u(0,y,t),~\psi_l^{(1)}(y,t)=\Psi_l^{(1)}\cdot u(0,y,t),\\
\varphi_l^{(2)}(x,t)=\Phi_l^{(2)}\cdot u(x,0,t),~\psi_l^{(2)}(x,t)=\Psi_l^{(2)}\cdot u(x,0,t).
\end{eqnarray*}
\quad Then at the corner
\begin{eqnarray*}
\dfrac{\partial \psi_1^{(1)}}{\partial y}(0,t)&=&+i\Psi_l^{(1)}k_2\left(1-R^{(2)}\right)\left(1+R^{(1)}\right)\exp(-i\omega t)\\
\dfrac{\partial \psi_1^{(1)}}{\partial t}(0,t)&=&-i\Psi_l^{(1)}\omega\left(1+R^{(2)}\right)\left(1+R^{(1)}\right)\exp(-i\omega t)\\
\dfrac{\partial \psi_m^{(2)}}{\partial t}(0,t)&=&-i\Psi_l^{(2)}\omega\left(1+R^{(2)}\right)\left(1+R^{(1)}\right)\exp(-i\omega t).
\end{eqnarray*}

}


% page4
\frame{
%\frametitle{}
\small


\quad Using 1D wave equations, we can impose
\begin{eqnarray*}
\Psi_l^{(1)}=\dfrac{\omega^2}{\omega^2-\alpha_l k_2^2},&~\Phi_l^{(1)}=\dfrac{\Psi_l^{(1)}-1}{\alpha_l}=\dfrac{k_2^2}{\omega^2-\alpha_l k_2^2}\\
\Psi_l^{(2)}=\dfrac{\omega^2}{\omega^2-\alpha_l k_1^2},&~\Phi_l^{(2)}=\dfrac{\Psi_l^{(2)}-1}{\alpha_l}=\dfrac{k_1^2}{\omega^2-\alpha_l k_1^2}
\end{eqnarray*}
\quad And from $u$ and all the $\psi$'s yield
$$k_n\cdot\dfrac{1-R^{(n)}}{1+R^{(n)}}=\omega\left(\gamma-\sum_{m=1}^{L}\dfrac{\beta_m}{\alpha_m}\Psi_m^{(n)}\right)~n=1,2,~\gamma=1+\sum_{m=1}^{L}\dfrac{\beta_m}{\alpha_m}$$
}


% page5
\frame{
%\frametitle{}
%$\Rightarrow$ 
\begin{eqnarray*}
\Rightarrow  k_n\left(1-R^{(n)}\right)&=&\omega\left(\gamma-\sum_{m=1}^{L}\dfrac{\beta_m}{\alpha_m}\Psi_m^{(n)}\right)\left(1+R^{(n)}\right)~ n=1,2\\
\Rightarrow~~~  \dfrac{\partial \psi_1^{(1)}}{\partial y}(0,t) &=&i\Psi_l^{(1)}k_2\left(1-R^{(2)}\right)\left(1+R^{(1)}\right)\exp(-i\omega t)\\ &=&i\omega\Psi_l^{(1)}\cdot\left(\gamma-\sum_{m=1}^{L}\dfrac{\beta_m}{\alpha_m}\Psi_m^{(2)}\right)\\
&&\qquad\qquad\left(1+R^{(2)}\right)\left(1+R^{(1)}\right)\exp(-i\omega t)
\end{eqnarray*}

}

% page6
\frame{
%\frametitle{}
\small
By $\omega^2=k_1^2+k_2^2$,
\begin{eqnarray*}
\Psi_l^{(1)}\cdot \Psi_m^{(2)}&=&\dfrac{\omega^2}{\omega^2-\alpha_m k_2^2}\cdot\dfrac{\omega^2}{\omega^2-\alpha_l k_1^2}\\
&=&\dfrac{\omega^2(\alpha_l+\alpha_m)-(k_1^2+k_2^2)\alpha_l\alpha_m}{(\omega^2-\alpha_l k_2^2)(\omega^2-\alpha_l k_1^2)}\cdot\dfrac{\omega^2}{\alpha_l+\alpha_m-\alpha_l\alpha_m}\\
&=&\dfrac{\alpha_l\omega^2-\alpha_l\alpha_m k_1^2+\alpha_m\omega^2-\alpha_l\alpha_mk_2^2}{(\omega^2-\alpha_l k_2^2)(\omega^2-\alpha_l k_1^2)}\cdot\dfrac{\omega^2}{\alpha_l+\alpha_m-\alpha_l\alpha_m}\\
&=&\left( \dfrac{\alpha_l}{\omega^2-\alpha_lk_2^2}+\dfrac{\alpha_m}{\omega^2-\alpha_lk_1^2}\right)\cdot\dfrac{\omega^2}{\alpha_l+\alpha_m-\alpha_l\alpha_m}\\
&=&\dfrac{\alpha_l}{\gamma_{m,l}}\Psi_l^{(1)}+\dfrac{\alpha_m}{\gamma_{m,l}}\Psi_l^{(2)}, \qquad\gamma_{m,l}=\alpha_l+\alpha_m-\alpha_l\alpha_m
\end{eqnarray*}

}

% page7
\frame{
%\frametitle{}
\small
\begin{eqnarray*}
\Rightarrow  \dfrac{\partial \psi_1^{(1)}}{\partial y}(0,t) &=&i\omega\Psi_l^{(1)}\cdot\left(\gamma-\sum_{m=1}^{L}\dfrac{\beta_m}{\alpha_m}\Psi_m^{(2)}\right)\\
&&\qquad\qquad\qquad\left(1+R^{(2)}\right)\left(1+R^{(1)}\right)\exp(-i\omega t)\\
&=&i\omega\left(\gamma\Psi_l^{(1)}-\sum_{m=1}^{L}\dfrac{\beta_m\alpha_l}{\alpha_m\gamma_{m,l}}\Psi_l^{(1)}-\sum_{m=1}^{L}\dfrac{\beta_m}{\gamma_{m,l}}\Psi_m^{(2)} \right)\\
&&\qquad\qquad\qquad\left(1+R^{(2)}\right)\left(1+R^{(1)}\right)\exp(-i\omega t)\\
&=&-\left(\gamma-\sum_{m=1}^{L}\dfrac{\beta_m\alpha_l}{\alpha_m\gamma_{m,l}}\right)\dfrac{\partial \psi_l^{(1)}}{\partial t}(0,t)+\sum_{m=1}^{L}\dfrac{\beta_m}{\gamma_{m,l}}\dfrac{\partial \psi_m^{(2)}}{\partial t}(0,t)
\end{eqnarray*}

}

% page8
\frame{
%\frametitle{}
\small
\quad Finally, 
\begin{block}{The Conner Condition}
\begin{eqnarray*}
\dfrac{\partial \psi_1^{(2)}}{\partial x}(x=0,t)+A_l\cdot\dfrac{\partial \psi_l^{(2)}}{\partial t}(y=0,t)-\sum_{m=1}^{L}C_{l,m}\cdot\dfrac{\partial \psi_m^{(1)}}{\partial t}(x=0,t)=0\\
\dfrac{\partial \psi_1^{(1)}}{\partial y}(y=0,t)+A_l\cdot\dfrac{\partial \psi_l^{(1)}}{\partial t}(x=0,t)-\sum_{m=1}^{L}C_{l,m}\cdot\dfrac{\partial \psi_m^{(2)}}{\partial t}(y=0,t)=0
\end{eqnarray*}
\qquad with $\displaystyle A_l=1+\sum_{m=1}^{L}\dfrac{\beta_m}{\alpha_m}-\sum_{m=1}^{L}\dfrac{\beta_m\alpha_l}{\alpha_m(\alpha_l+\alpha_m-\alpha_m\alpha_l)}$ \\
\qquad\qquad and $C_{l,m}=\dfrac{\beta_m}{\alpha_l+\alpha_m-\alpha_m\alpha_l}$.
\end{block}


}

%% page9
%\frame{
%\frametitle{}
%\small
%
%}


%%%%%%%%%%%%%%%%%%%%%%%%%%%%%%%%%%%%%%%%%%%%%%%%%%%%%%%%%%
%%%%%%%%%%%%%%%%%%%%%%% Section 4 %%%%%%%%%%%%%%%%%%%%%%%%
%%%%%%%%%%%%%%%%%%%%%%%%%%%%%%%%%%%%%%%%%%%%%%%%%%%%%%%%%%
\section{A Numerical Scheme}


%%%%%%%%%%%%%%%%%%%%%% subsection 4-1 %%%%%%%%%%%%%%%%%%%%%%%%%
\subsection{}
% page1
\frame{
%\frametitle{}
\small

\begin{center}
\includegraphics[scale=0.7]{grid2.jpg}
\quad
\includegraphics[scale=0.2]{point1.jpg}
\end{center}
%\begin{eqnarray*}
%\text{A : X} &\rightarrow& \left(\varphi_m^{(2)}\right)_{i}^n\\
%\text{B : }\blacklozenge &\rightarrow& \left(\varphi_m^{(1)}\right)_{j}^n\\
%\text{D : }\blacktriangle &\rightarrow& \left(\psi_m^{(1)}\right)_{1/2}^n\\
%\text{E : }\bigcirc &\rightarrow& \left(\varphi_m^{(2)}\right)_{1/2}^n\\
%\text{otherwise, }& u_{i,j}^n&
%\end{eqnarray*}
\quad A grid is shifted in such a way that the boundary of $\Omega$ is centered between the two last lines of nodes.\\

}

% page2
\frame{
%\frametitle{}
\small
\quad Using the second order scheme in both space and time, wave equation is descrived :
$$\dfrac{u_{i,j}^{n+1}-2u_{i,j}^{n}+u_{i,j}^{n-1}}{\Delta t^2} - \dfrac{u_{i+1,j}^{n}-2u_{i,j}^{n}+u_{i-1,j}^{n}}{\Delta x^2} -
\dfrac{u_{i,j+1}^{n}-2u_{i,j}^{n}+u_{i,j-1}^{n}}{\Delta y^2}$$
\qquad,where $(i,j)$ lie in the set of all semi integer.\\
\vspace{10}
\quad With this formula, one compute all the $u_{i,j}^{n+1}$ with the exception of those corresponding to nodes on the two extra-lines $x_1=\dfrac{\Delta x}{2},~y_1=\dfrac{\Delta y}{2}$.
\newline\newline
I Don't Know...
}


% page3
\frame{
%\frametitle{}
\small

}



%% page4
%\frame{
%\frametitle{Linear System}
%\small
%\ For a linear system $u_t+Au_x=0$, $u^{*}(U_{j}^n,U_{j+1}^n)$ can be written as
%$$u^{*}(U_{j}^n,U_{j+1}^n)=U_{j}^n+\sum_{\lambda_i <0}\alpha_i r_i =U_{j+1}^n-\sum_{\lambda_i >0}\alpha_i r_i $$
%where $Ar_i=\lambda_ir_i$ and $U_{j+1}^n-U_{j}^n=\sum{\alpha_ir_i}=R\alpha$.\newline\newline
%$\begin{array}{lllll}
%F(U_{j}^n,U_{j+1}^n)&=&f(u^{*}(U_{j}^n,U_{j+1}^n)) & & \\
%&=&Au^{*}(U_{j}^n,U_{j+1}^n)&&\\
%&=&\displaystyle AU_{j}^n+\sum_{\lambda_i <0}\alpha_i\lambda_i r_i&=&\displaystyle AU_{j+1}^n-\sum_{\lambda_i >0}\alpha_i\lambda_i r_i\\
%&=&\displaystyle AU_{j}^n+R\Lambda^{-}\alpha&=&\displaystyle AU_{j+1}^n-R\Lambda^{+}\alpha\\
%&=&\displaystyle AU_{j}^n+A^{-}\left( U_{j+1}^n-U_{j}^n\right)&=&\displaystyle AU_{j+1}^n-A^{+}\left( U_{j+1}^n-U_{j}^n\right)\\
%\end{array}$
%
%
%}
%
%
%
%% page5
%\frame{
%\frametitle{Linear System}
%\small
%$\circ$ $\left\{
%\begin{array}{lllll}
%F(U_{j}^n,U_{j+1}^n)&=&f(u^{*}(U_{j}^n,U_{j+1}^n))&=&AU_{j}^n+A^{-}\left( U_{j+1}^n-U_{j}^n\right)\\
%F(U_{j-1}^n,U_{j}^n)&=&f(u^{*}(U_{j-1}^n,U_{j}^n))&=&AU_{j}^n-A^{+}\left( U_{j}^n-U_{j-1}^n\right)
%\end{array}$
%
%\begin{block}{Godunov's Method for the Linear System}
%\begin{eqnarray*}
%U_j^{n+1}&=&U_j^n-\dfrac{k}{h}\left[f(u^{*}(U_{j}^n,U_{j+1}^n))-f(u^{*}(U_{j-1}^n,U_{j}^n))\right]\\
%&=&U_j^n-\dfrac{k}{h}\left[\dfrac{k}{h}A^{-}(U_{j+1}^n-U_{j}^n)+A^{+}(U_j^n-U_{j-1}^n)\right]
%\end{eqnarray*}
%\end{block}
%$\circ$ This is the upwind method for the linear system.
%}
%
%
%% page6
%\frame{
%\frametitle{Entropy Condition}
%\small
%\ Let $\eta(u)$ be a convex entropy function and let $\psi(u)$ be a entropy flux.\\
%\ Suppose that $\tilde{u}^n$ satisfies the entropy inequality.\\
%
%$$\displaystyle \frac{1}{h}\int_{x_{j-\frac{1}{2}}}^{x_{j+\frac{1}{2}}}\eta(\tilde{u}^n(x,t_{n+1}))dx\le\frac{1}{h}\int_{x_{j-\frac{1}{2}}}^{x_{j+\frac{1}{2}}}\eta(\tilde{u}^n(x,t_n))dx\\
%\qquad\qquad\qquad\displaystyle - \frac{1}{h} \left[\int_{t_n}^{t_{n+1}}\psi(\tilde{u}^n(x_{j+\frac{1}{2}},t))dt- \int_{t_n}^{t_{n+1}}\psi(\tilde{u}^n(x_{j-\frac{1}{2}},t))dt \right]$$\newline\newline
%
%
%Since $\tilde{u}^n$ is constant along $t=t_n,x=x_{j-\frac{1}{2}}$ and $x=x_{j+\frac{1}{2}}$,
%
%$$\displaystyle \frac{1}{h}\int_{x_{j-\frac{1}{2}}}^{x_{j+\frac{1}{2}}}\eta(\tilde{u}^n(x,t_{n+1}))dx\le\eta(U_j^n) - \frac{k}{h} \left[ \psi(u^{*}(U_{j}^n,U_{j+1}^n))-\psi(u^{*}(U_{j-1}^n,U_{j}^n)) \right].$$
%
%
%}
%
%
%% page7
%\frame{
%%\frametitle{Entropy Condition}
%%\small
%%Since $\eta$ is convex, by \emph{Jensen's inequality},\\
%%$$\displaystyle\eta\left(\frac{1}{h}\int_{x_{j-\frac{1}{2}}}^{x_{j+\frac{1}{2}}}\tilde{u}^n(x,t_{n+1})dx\right)\le\frac{1}{h}\int_{x_{j-\frac{1}{2}}}^{x_{j+\frac{1}{2}}}\eta(\tilde{u}^n(x,t_{n+1}))dx$$
%%
%%If the numerical entropy flux $\Psi(U_{j}^n,U_{j+1}^n)=\psi(u^{*}(U_{j}^n,U_{j+1}^n))$, then
%%
%%
%%$$\displaystyle \eta(U_j^{n+1})\le\eta(U_j^n) - \frac{k}{h} \left[ \Psi(U_{j}^n,U_{j+1}^n)-\psi(U_{j-1}^n,U_{j}^n) \right].$$\\
%%$\therefore$ By Godunov's method, we can obtain \emph{entropy-satisfying solution}.
%}



%%%%%%%%%%%%%%%%%%%%%% Reference %%%%%%%%%%%%%%%%%%%%%%%%%
\section{ }
\subsection{ }
\frame{
\large
\frametitle{Reference}
\small

1. Antoine, X., Arnold, A., Besse, C., Ehrhardt, M., and Schädle, A.: ‘A review of transparent and artificial boundary conditions techniques for linear and nonlinear Schrödinger equations’, 2009 \newline\newline
2. Collino, F.: ‘ High order absorbing boundary conditions for wave propagation models: straight line boundary and corner cases’ (1993, edn.), pp. 161-171\newline\newline
3. Engquist, B., and Majda, A.: ‘ Absorbing boundary conditions for numerical simulation of waves’, Proceedings of the National Academy of Sciences, 1977, 74, (5), pp. 1765-1766\newline\newline
4. Givoli, D.: ‘High-order local non-reflecting boundary conditions: a review’, Wave Motion, 2004, 39, (4), pp. 319-326\newline\newline



}
%%%%%%%%%%%%%%%%%%%%%% Thank you %%%%%%%%%%%%%%%%%%%%%%%%%
\section{ }
\subsection{ }
\frame{
\large
\frametitle{}
\ $$\textbf{Thank you!}$$
}




\end{document}
